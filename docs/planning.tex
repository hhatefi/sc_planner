\documentclass[a4paper]{article}

\usepackage{amsmath}
\usepackage{amsfonts}
\usepackage{mathtools}

\newcommand{\nat}{\mathbb{N}}
\newtheorem{problem}{Problem}
\newtheorem{definition}{Definition}
\newcommand{\pow}[1]{2^{#1}}
\newcommand{\dotcup}{\mathbin{\dot{\cup}}}
\newcommand{\ands}{\mathcal{A}}
\newcommand{\ors}{\mathcal{O}}
\newcommand{\SCP}{\textsl{SCP}}
\newcommand{\SC}{\textsl{SC}}
\newcommand{\chain}{\mathit{ch}}

\begin{document}
\author{Hassan Hatefi}
\title{Supply Chain Planning}
\date{\today}
\maketitle

This document explains the algorithm to solve supply chain planning
problem. At first, the problem is defined and then it is formulated by
a mixed integer linear program. At the end, its complexity is briefly
discussed.

Throughout this document, the power set of set $P$ is referred to as
$\pow{P}$, which is the set of all subsets of $P$, i.e. $\{P'\mid
P'\subseteq P\}$. Disjoint union is denoted by $\dot{\cup}$. The
definition of \emph{supply chain} is as follows.
\begin{definition}[Supply Chain]\label{def:sc}
  Let $P$ be a set of \emph{products} with $o:P\to\nat$ as its order
  function such that $o(p)$ is the number of ordered product $p\in
  P$. Moreover, $C$ is a set of \emph{components} with stock function
  defined by $s:C\to\nat_0$. A product or a component is called
  \emph{entity} and the set of all entities $\{P\dotcup C\}$ is
  referred to as $E$. Let $C_s\subseteq C$ the component that are
  directly supplied by a supplier. Each product or component is
  chained to the network by function $\chain:E\to\pow{C}$. Then
  $\SC=(E=\{P\dotcup C\},o,s,\chain,C_s)$ is a supply chain.
\end{definition}
Chain function $\chain$ specifies how an entity is connected to other entities
in the chain. Formally speaking, entity $e$ is directly supplied by all
entities in $\chain(e)$. The type of supply is recognized by partitioning
entities into sets $\ands$ and $\ors$ corresponding \emph{and} and \emph{or}
delivery, respectively. The set of entities that are suppied by entity $e$ is
denoted by $\delta(e)$, i.e.~$\{e'\mid e\in\chain(e)\}$.

Entities can flow through the chain down from suppliers and inventories as the
source and be delivered as the products. An assignment of the flows to the
chain that satisfies some constraints is called a \emph{plan}.

\begin{definition}[Plan]\label{def:plan}
  A \emph{plan} is an assignment of flows going out of suppliers ($s_c$ for
  $c\in C_s$), going out of inventories ($v_c$ for $c\in C$), passing through
  edges $y_{e,f}$ for all $e\in E$, for all $f\in\chain(e)$ and being delivered
  as an entity ($x_e$ for $p\in P$) that satisfy the following constrains
  \begin{align}
    0\le x_p\le o(p),&\qquad\forall~p\in P\label{eq:prod}\\
    0\le v_c\le s(c),&\qquad\forall~c\in C\\
    x_e=y_{e,f},&\qquad\forall~e\in\ands, \forall~f\in\chain(e)\\
    x_e\le
    y_{e,f}+(1-z_{e,f})\bar{x}_{e},&\qquad\forall~e\in\ors.~\forall~f\in\chain(e)\label{eq:or:ub}\\
    x_e\ge
    y_{e,f}-(1-z_{e,f})\bar{x}_{e},&\qquad\forall~e\in\ors.~\forall~f\in\chain(e)\label{eq:or:lb}\\
    \sum_{f\in\delta(c)}y_{c,f}=x_c+s_c+v_c,&\qquad\forall~c\in C_s\\
    \sum_{f\in\delta(c)}y_{c,f}=x_c+v_c,&\qquad\forall~c\in C\setminus C_s\\
    \sum_{f\in\chain(e)}z_{e,f}=1,&\qquad\forall~e\in\ors\\
    z_{e,f}\in\{0,1\},&\qquad\forall~e\in\ors\\
    s_c\ge 0,&\qquad\forall~c\in C_s\label{eq:src}
  \end{align}
\end{definition}
In Ineqs.~\eqref{eq:or:ub} and~\eqref{eq:or:lb}, term $\bar{x}_e$ refers to
any upper bound on the flow delivered to entity $e$. It can be easily computed
by a bachward search through the chain.
 
I define two problems related to supply chain. \emph{Supply Chain
  Planning} problem is a decision problem to check whether a supply
chain can satisfy its order.
\begin{problem}[Supply Chain Planning (SCP)]\label{prob:scp}
  Given a supply chain \SC{} as defined by Def.~\ref{def:sc}, \SCP{} decides
  if there is a plan satisfying $x_p=o(p)$ for all $p\in P$.
\end{problem}
The optimization version of Prob.~\ref{prob:scp} is defined next.
\begin{problem}[Maximal Supply Chain Plan~(MSCP)]\label{prob:mscp}
  Given a supply chain \SC{} as defined by Def.~\ref{def:sc} and an affine
  function $r$ on flow variables defined in described in
  Def.~\ref{def:plan}, MSCP is the following optimization problem:
  \begin{align*}
    &\max~r(x,y,v,s)\\
\mathrm{subject}\;\mathrm{to}&\\
&\mathrm{constrains}~\eqref{eq:prod}~\mathrm{to}~\eqref{eq:src}
  \end{align*}
\end{problem}
It is straightforward to prove Prob.~\ref{prob:scp} is
NP-complete. One can easily encode subset sum problem as \SCP{}
problem.

\end{document}
